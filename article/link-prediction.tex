\newcommand{\score}[1]{{\mathrm {#1}}}

\section{Link Predicton}

Link prediction in a classic and well-studied problem over social
networks, e.g.~\cite{}. Formally, the {\em link prediction problem\/}
is: Given a snapshot of a social network at a given point in time,
predict which edges will be added to the network as it naturally
evolves. The ability to predict such links is considered important for
friend recommendation, i.e., as predicted links are likely to
represent natural relationships in the network. Link prediction is
also useful for inferring missing links, i.e., discovering which links
are surprisingly missing (and then the social process leading to this
formation can be further analyzed). 

Many score functions have been developed for predicting links. In general, a
a score function is defined over pairs of nodes, and indicates
the likelihood of a link forming between these nodes. As of now, no
clear winner for the link prediction problem has been found. Instead,
different score functions have proven useful over varying datasets. As we will
see later, score functions may differ substantionally in their ease of
implementation and in their efficiency.

Below we detail the score functions considered in this paper. These particular
functions were chosen for two reasons. First, they are  the
``bread-and-butter'' of link prediction, i.e., they are
widely considered and studied. Second, their implementation requires a
variety of types of accesses to the social network graph (only to
neighbors, iterating over paths or random walks). Hence, they can
perhaps shed light on how efficient other common graph operations
would be over social networks (in different system implementations). 


In total, we considered seven score functions for link prediction. In the
following, we assume that $x$ and $y$ are nodes in a social
network. We use $N(x)$ to denote the set of all neighbors of
$x$. Finally, we
use $P^l(x,y)$ to denote the set of all paths of length $l$ between $x$
and $y$.
\begin{itemize}
\item {\bf Common neighbors:} The common neighbors $\score{cneigh}$
  function measures the number of neighbors in common to $x$ and $y$,
  i.e., \[ \score{cneigh}(x,y) = |N(x) \cap N(y)| \,.\]
\item {\bf Jaccard coefficient:} The Jaccard coefficient is slightly
  more sophisticated than common neighbors, as it also takes into
  considering the total number of neighbors that $x$ and $y$
  have. This is significant, since nodes with many neighbors are more
  likely to have neighbors in common. Formally, the Jaccard
  coefficient is defined as
\[ \score{jacc}(x,y) = \frac{|N(x) \cap N(y)|}{|N(x) \cup N(y)|}\,.\]
\item {\bf Adamic-Adar:} The Adamic-Adar measure~\cite{} considers
  nodes to be related if they have common neighbors, but weights the
  importance of a common neighbor $z$ proportionately to the number of
  neighbors of $z$. This gives more weight to rarer common neighbors,
  i.e., to nodes $z$ that have few neighbors themselves, as follows
\[ \score{aa}(x,y) = \sum_{z\in N(x)\cap N(y)} \frac{1}{\log{|N(z)|}}\,.\]
\item {\bf Preferential attachment:} In this score function, popular
  nodes (i.e., those with many neighbors) are considered more likely
  to form relationships one with another
\[ \score{pref}(x,y) = |N(x)|\times |N(y)|\,.\]
\item {\bf Graph distance:} Unlike all previous measures, that only
  directly considered the neighbors of $x$ and $y$, graph distance
  takes into consideration the relationship between nodes that may be
  far away. Thus, this scoring function $\score{dist}(x,y)$ is simply the
  the length of the shortest path between  $x$ and $y$. 
\item {\bf Katz measure:} The Katz measure~\cite{} takes into
  consideration all paths between $x$ and $y$, but these paths are
  weighted by their length. In general, $\score{katz}$ is computed as
\[ \score{katz}(x,y) = \sum_{l=1}^\infty \beta^l |P^l(x,y)|\,,\]
where $0<\beta<1$. We used a $\beta$ value of 0.1. For efficiency
purposes, we only summed until paths 
of length 3.
\item {\bf Rooted PageRank:} PageRank~\cite{} computes a general importance
  value for a node by measuring the stationary distribution weight of
  a node $x$ under a random walk, where with probability $\alpha$ one
  jumps to a random node in the graph, while with probability
  $1-\alpha$ one goes to a random neighbor of the current node. Rooted
  PageRank adapts this measure to compute the importance of a node $y$
  with relationship to a node $x$. Thus, $\score{rPR}(x,y)$ computes
  the stationary distribution weight of   a node $y$ under a random
  walk, where with probability $\alpha$ one jumps back to $x$, while
  with probability $1-\alpha$, one proceeds as before (i.e., to a
  random neighbor). In our implementation we chose $\alpha$ \footnote{Sara: rename alpha to d? (damping factor)} as
  0.85 \footnote{Sara: for the damping factor value, should we cite? http://infolab.stanford.edu/pub/papers/google.pdf} 
\end{itemize}
Note that for all measures other than graph distance, a
higher value for the scoring function indicates greater likelihood of
link formation, while for graph distance, the opposite is
true. 
